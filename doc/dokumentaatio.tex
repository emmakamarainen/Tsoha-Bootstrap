\documentclass[a4paper]{article}
\usepackage[utf8]{inputenc}
\usepackage[T1]{fontenc}
\usepackage[finnish]{babel}
\usepackage{geometry}
\usepackage{amsmath}
\usepackage{amsthm}
\usepackage{amssymb}
\begin{document}

\section*{Tietokantasovellus, harjoitustyö}
\section{Johdanto}


%Johdantoon kirjoitetaan lyhyt, ytimekäs kuvaus siitä, mikä on työn aihe, mitä työllä kuuluisi pystyä tekemään ja mitä tekniikoita siinä käytetään.
%
%Järjestelmän tarkoitus
%Tiivis kuvaus siitä mistä on kyse.
%Millaisen toiminnan tukemiseen järjestelmä on tarkoitettu.
%Mitkä ovat järjestelmän tavoitteet.
%Nämä tiedot saa yleensä tehtäväkuvauksesta, kirjoita kuitenkin omin sanoin.

%Toteutus-/toimintaympäristö
%Missä ympäristössä työ toteutetaan (yleensä laitoksen users-palvelimella Tomcat- tai Apache-palvelimen alla)
%Täytyykö web-sovelluksen alustajärjestelmän tukea jotain tiettyä ohjelmointikieltä. (esim. Java, Ruby, PHP..?)
%Jos edellytetään jotain sovelluskehystä tulisi sekin mainita.
%Täytyykö käyttäjän selaimen tukea jotain tiettyä ohjelmointikieltä (esim. javascript?).
%Edellyttääkö ohjelmisto jonkun tietyn tietokannan käyttöä vai voiko sitä vaihtaa helposti. Useimmat työt toimivat vain yhdellä kannalla.

\end{document}